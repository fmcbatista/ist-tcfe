\section{Simulation Analysis}
\label{sec:simulation}

\subsection{Operating Point Analysis}

Following our theoretical analysis, the circuit was simulated in the Ngspice software. 
In step (1), we were asked to simulate the operating point for t<0, in order to find the voltages in all nodes and the currents in all branches. The results are shown in Table~\ref{tab:alinea1}.

\begin{table}[h]
  \centering
  \begin{tabular}{|l|r|}
    \hline    
    {\bf Name} & {\bf Value [A or V]} \\ \hline
    \input{op_TAB1}
  \end{tabular}
  \caption{Operating point for t<0. Simulated Values for voltage (V) and current (A) using Ngspice.}
  \label{tab:alinea1}
\end{table}

Then, in step (2), we simulated the operating point considering that the voltage of the source in the instant t=0 was null and replacing the capacitor with a voltage source that was equal to V6-V8, being this values the voltages in the capacitor's nodes that were obtained in (1). This step was needed because... The results are shown in Table~\ref{tab:alinea2}.

\begin{table}[h]
  \centering
  \begin{tabular}{|l|r|}
    \hline    
    {\bf Name} & {\bf Value [A or V]} \\ \hline
    \input{op_TAB2}
  \end{tabular}
  \caption{Operating point for Vs=0 (instant t=0). Simulated Values for voltage (V) and current (A) using Ngspice.}
  \label{tab:alinea2}
\end{table}

In step (3), we simulated the natural response of the circuit. In Figure~\ref{fig:plot(3)} we can find the plot of v6 in the interval [0, 20]ms.

\begin{figure}[h] \centering
\includegraphics[width=0.8\linewidth]{forced.eps}
\caption{Plot of v6(t) in the interval [0, 20]ms.}
\label{fig:plot(3)}
\end{figure}

Later, in step (4), we were asked to simulate the natural and forced response on node 6 by repeating step (3) with vs(t) as given in Equation~\ref{eq:vs} and f=1kHz. In Figure~\ref{fig:plot(4)} we can see the plot of both the stimulus and the response.

\begin{equation}
  V_{s}(t) = sin(2 \pi f t),
  \label{eq:vs}
\end{equation}

\begin{figure}[h] \centering
\includegraphics[width=0.8\linewidth]{forced.eps}
\caption{Plot of v6(t) in the interval [0, 20]ms.}
\label{fig:plot(4)}
\end{figure}




