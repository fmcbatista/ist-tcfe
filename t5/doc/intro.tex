\section{Introduction}

The main goal of this laboratory assignment was to implement a band pass filter, that would have a gain of 40dB and a central frequency of 1khz. A band pass filter is a circuit that must be designed in a way that blocks all frequencies, except the desired ones belonging in a certain range. This task is performed with the aid of an OP-AMP. This amplifier consists of a transistor based device, which has some particular traits such as high voltage gain, high input impedance and low output impedance.

In the figure below we can see a representation of the circuit studied in this laboratory:

%\begin{figure}[h] \centering
%\includegraphics[width=0.8\linewidth]{circfig.pdf}
%\caption{Circuit.}
%\label{fig:circ}
%\end{figure}

We then proceed to the simulation and the theoretical analysis, where we make some adjustments in an attempt to improve several parameters, that are then reflected in a merit figure.

Also, the values used for the resistors and capacitors are the following:

\begin{table}[h]
  \centering
  \begin{tabular}{|l|r|}
    \hline    
    {\bf Component} & {\bf Value} \\ \hline
    \input{valintro.txt}
  \end{tabular}
  \caption{Values of resistors and capacitors.}
  \label{tab:valintro}
\end{table}
